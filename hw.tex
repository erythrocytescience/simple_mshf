\documentclass{article}

% use packages
\usepackage[T2A]{fontenc}
\usepackage[russian]{babel}
\usepackage{graphicx} % Required for inserting images
\usepackage[a4paper,top=2cm,bottom=2cm,left=3cm,right=3cm,marginparwidth=1.75cm]{geometry}
\usepackage{authblk} % Для аффилиаций
\usepackage{datetime} % Для форматирования даты
\usepackage[colorlinks=true, allcolors=blue]{hyperref} % Для гиперссылок и \phantomsection
\usepackage{url} % Для форматирования ссылок в библиографии
\usepackage{amsmath}
\usepackage{xcolor}
\usepackage{array}
\usepackage{multirow}
% \usepackage[table]{xcolor} % можно закрасить заголовок столбцов таблицы
\usepackage{hyphenat}    % для переноса слов
\usepackage{ragged2e}    % для \RaggedRight (лучше, чем \raggedright)
\usepackage{float} % Добавьте в преамбулу
\usepackage{svg}
\usepackage{subcaption}

% =================== SETTINGS ==============================
\renewcommand\Authand{, }
\renewcommand\Authands{, }

% Keywords command
\providecommand{\keywords}[1]
{
  \small
  \textbf{\textit{Ключевые слова: }} #1
}

\newcommand{\mysub}[1]{%
  % \par\noindent{\large\bfseries\underline{#1}}\par\vspace{0.5em}
  \par\vspace{0.5em}\noindent{\normalsize\underline{#1}}\par\vspace{0.5em}%
}

\captionsetup[figure]{justification=centering, singlelinecheck=false}

\renewcommand{\thesubfigure}{\asbuk{subfigure}}
\captionsetup[subfigure]{labelformat=brace}
% =================== SETTINGS END ==============================

\title{АНАЛИТИЧЕСКАЯ МОДЕЛЬ ДЛЯ РАСЧЕТА ПРИТОКА ЖИДКОСТИ
К ГОРИЗОНТАЛЬНОЙ СКВАЖИНЕ С МНОГОЗОННЫМ ГИДРОРАЗРЫВОМ ПЛАСТА}
\author[1]{Мазо А.Б.}
\author[1]{Поташев К.А.}
\author[2]{Хамидуллин М.Р.}
\affil[1]{Казанский федеральный университет, Казань, Россия}
\affil[2]{Научно-исследовательский центр ``Курчатовский институт''}
\date{19 июля 2025~--~\today}

\begin{document}

\maketitle

\tableofcontents

\listoffigures

\listoftables.

\begin{abstract}
  Введите здесь текст аннотации
\end{abstract}

\section*{Введение}
\addcontentsline{toc}{section}{Введение} % чтобы отображалось в содержании (текст должен совпадать)

\section{Постановка задачи}

Рассмотрим задачу о притоке однофазной жидкости к горизонтальной скважине (ГС),
простимулированной многозонным гидравлическим разрывом пласта (МГРП)
с трансверсальными трещинами конечной проводимости. Примем следующую схему
ГС с МГРП (рис.~\ref{fig:kham_main_scheme}).

Введем декартовую систему координат $xyz$ с вертикальной осью $z$ и осью $y$,
направленной вдоль ствола скважины радиуса $r_w$ и длины $L$.
Кровля и подошва пласта горизонтальны, толщина пласта постоянна и равна $2H$.
Абсолютную проницаемость пласта обозначим через $k=k\left(x,y,z\right)$,
а удельную (на единицу толщины пласта)
гидропроводность~--~$\sigma = k / \mu$, где $\mu$~--~вязкость флюида.

Контур питания в плоскости $xy$ представим прямоугольником, размеры которого
определяются системой разработки залежи. Стенки трещин представим парой
прямоугольных плоскостей размерами $2H \times 2h$ с расстоянием между
ними $2\delta$ (ширина раскрытия трещины). Трещины МГРП имеют проницаемость
$k_f$ и расположены на расстоянии $2d$ друг от друга. Соответствующая удельная гидропроводность трещины равна $\sigma_f$.

\begin{figure}[!ht]
  \centering
  \includesvg[width=0.8\linewidth]{images/schemes/kham_3D_scheme.svg}
  \caption{\label{fig:kham_main_scheme}Схема горизонтальной скважины с МГРП}
\end{figure}

Примем следующие упрощения: однофазное фильтрационное течение стационарно
и определяется разностью давления $p_w$ на скважине и пластового давления
$p_r$ на внешнем контуре; капиллярные и гравитационные силы не учитываются;
падением давления вдоль ствола скважины пренебрегаем.

Математическая модель фильтрации в безразмерных переменных
\begin{equation}
  \displaystyle
  \begin{gathered}
    \bar{x},\bar{y},\bar{z} = \dfrac{x,y,z}{R}, \quad
    \bar{p} = \dfrac{p - p_w}{p_r - p_w}, \quad
    \bar{\sigma} = \dfrac{\sigma}{\sigma_0}, \quad
    \bar{u}=\dfrac{u}{u_0},  \quad
    \bar{q} = \dfrac{q}{q_0}    \\
    u_0 = \sigma_0 \dfrac{p_r - p_w}{R}, \quad
    \sigma_0 = \dfrac{k_0}{\mu}, \quad
    q_0 = u_0 R^2
  \end{gathered}
  \label{eq:kham_dimless}
\end{equation}
сводится  к уравнениям  давления $p$ в коллекторе и $p_f$ в трещинах
гидроразрыва~\cite{lit:kham_mazo_uzku_2015}
\begin{equation}
  \displaystyle
  - \operatorname{div} \vec{u} = 0,
  \quad u=-\sigma \text{grad} p;
  \label{eq:kham_main_p_res}
\end{equation}
\begin{equation}
  \displaystyle
  \begin{gathered}
    \triangle p_f + \dfrac{\sigma}{2M}\left.\dfrac{\partial p}{\partial y} \right|_{y_f - \delta}^{y_f + \delta} = 0, \\[8pt]
    \quad M = \dfrac{C_f I_h}{2}, \quad C_f = \dfrac{2 k_f \delta}{k h}, \quad I_h=\dfrac{h}{R}
  \end{gathered}
  \label{eq:kham_main_p_fract}
\end{equation}
где $k_0$~--~характерный масштаб проницаемости пласта, $C_f$~--~безразмерная проводимость трещины, $I_h$~--~степень
вскрытия~\cite{lit:kham_valko_economides_2001}.

Граничные условия таковы: кровля и подошва пласта, торцы трещин
непроницаемы. На боковых поверхностях трещин задается равенство давлений и
нормальных компонент скорости. Аналогичные условия сопряжения ставятся и на
боковой поверхности $r=\sqrt{x^2 + z^2} = r_w$ скважины: на ее перфорированных участках задается давление $p_w$, а на неперфорированных~--~условие изоляции; на контуре питания задается постоянное давление $p_r$.

\section{Упрощенная методика расчета дебита ГС с МГРП}

Задача (\ref{eq:kham_main_p_res}),~(\ref{eq:kham_main_p_fract}) может быть
решена численно, однако, такой подход требует значительных вычислительных затрат.
Ниже предлагается упрощенная методика, основанная на учете различных видов симметрии фильтрационного течения в окрестности ГС с МГРП. При этом прямое численное моделирование применяется для уточнения и проверки приближённого решения.
Постановка задачи и формулы для расчета притока жидкости к горизонтальной
скважине без трещин МГРП получаются аналогично работе~\cite{lit:kham_mazo_uzku_2015},
с отличием лишь в масштабирующих коэффициентах. За подробными выводами можно обратиться к указанному источнику;
в данной работе мы приведём только конечные формулы и сосредоточимся
на более детальном анализе области их применимости.

% ==========================================================================
% ==========================================================================
% ====================== ГОРИЗОНТАЛЬНАЯ СКВАЖИНА ===========================
% ==========================================================================
% ==========================================================================

\subsection{Приток к горизонтальной скважине без ГРП}

% ====================== ОСНОВНОЙ СТВОЛ ===========================
\subsubsection{Основной ствол скважины}

Рассмотрим задачу о притоке однофазного флюида к ГС без трещин МГРП. Будем считать пласт однородным. Пренебрегая падением давления вдоль ствола скважины, сведём
задачу к двумерной в сечении при $y = y_0 \in \left(0, L\right)$
(рис.~\ref{fig:kham_hw_inner_scheme}). Отметим, что плоский характер течения
нарушается, когда $y_0$ приближается к торцам скважины $y = 0$ и $y = L$; влияние концевых
эффектов будет рассмотрено отдельно.

\begin{figure}[!ht]
  \centering
  \includesvg[width=0.5\linewidth]{images/schemes/kham_hw_inner_scheme.svg}
  \caption{Область решения задачи о притоке жидкости к основному участку ствола ГС}
  \label{fig:kham_hw_inner_scheme}
\end{figure}

Покажем, что в подобласти (II), $H \leq x \leq 1$ (рис.~\ref{fig:kham_hw_inner_scheme}), течение является плоско-параллельным.
При этом расход и давление определяются как
\begin{equation}
  \displaystyle
  q^{II} = \dfrac{H \left(1-p_{\Gamma}\right)}{1-H}, \;\;\;
  p^{II}(x) = p_{\Gamma} + (1 - p_{\Gamma})\dfrac{x - H}{1-H}
  \label{eq:kham_hw_inner_par_qp}
\end{equation}
где~$p_{\Gamma}$~--~подлежащее определению давление при $x=H$.

Будем далее считать, что на окружности
$\Gamma=\left\{\left(x,\, z\right): \sqrt{x^2 + z^2} = H \right\}$
давление близко к постоянному и равно $p_{\Gamma}$ при $x=H$.
Тогда в подобласти (I), ограниченной окружностями $\gamma$ и $\Gamma$ и
лучами $x = 0$ и $z = 0$, фильтрация будет плоско-радиальной
\begin{equation}
  \displaystyle
  q^I = \theta \dfrac{\pi p_{\Gamma}}{2 \ln{\left(H/r_w\right)}}, \;\;\;
  p^I(r) = p_{\Gamma} \dfrac{\ln{\left(r/r_w\right)}}{\ln{\left(H/r_w\right)}}.
  \label{eq:kham_hw_inner_rad_qp}
\end{equation}
Здесь $\theta$~--~поправочный коэффициент, значение которого определяется
при сопоставлении $q^I$ из (\ref{eq:kham_hw_inner_rad_qp}) с ``точным''
значением расхода, полученным с помощью численного решения. Поправочный
коэффициент вводится из-за того, что мы снесли давление $p_{\Gamma}$ с прямой $x= H$
на окружность $\Gamma$ радиуса $H$.

Приравняв значения расходов $q^I$ из (\ref{eq:kham_hw_inner_rad_qp})
и $q^{II}$ (\ref{eq:kham_hw_inner_par_qp}), получим
\begin{equation}
  \displaystyle
  p_{\Gamma} = \dfrac{2 H \ln{\left(H/r_w\right)}}{2 H \ln{\left(H/r_w\right)
  + \pi \theta \left(1 - H\right)}}.
  \label{eq:kham_hw_inner_pg}
\end{equation}
Для завершения описания модели необходимо определить итоговое распределение
давления во всей области. После вычисления давления на
границе $p_{\Gamma}$ по формуле (\ref{eq:kham_hw_inner_pg}) и получения
выражений для давления в подобластях, итоговое распределение давления во всей
области определяется кусочно-заданной функцией, непрерывной на границе $\Gamma$:
\begin{equation}
  p =
  \begin{cases}
    p_{\Gamma} \dfrac{\ln{\left(r/r_w\right)}}{\ln{\left(H/r_w\right)}}, & r_w \leq r \leq H, \\
    p_{\Gamma} + (1 - p_{\Gamma})\dfrac{x - H}{1-H}, & H \leq x \leq 1.
  \end{cases}
  \label{eq:kham_hw_inner_total_pressure}
\end{equation}
Полный приток к стволу скважины с учетом~(\ref{eq:kham_hw_inner_rad_qp})
\begin{equation}
  \displaystyle
  Q_w^P = \dfrac{2\pi L p_{\Gamma}}{\ln{\left(H/r_w\right)}}
  \label{eq:kham_hw_inner_Q}
\end{equation}

Поправочный коэффициент $\theta$ определяется из равенства двух дебитов —
аналитического $Q_w^P$ и численного $Q_{\text{num}}$
\begin{equation}
  \displaystyle
  \theta = \dfrac{H}{1-H} \left(\dfrac{4L}{Q_{\text{num}}}
  - \dfrac{2}{\pi} \ln \left[\dfrac{H}{r_w}\right]\right),
  \label{eq:kham_hw_inner_theta}
\end{equation}
где $Q_{\text{num}}$~--~дебит, полученный из численного решения аналогичной задачи. Для его вычисления будем использовать MRST~\cite{lit:kham_mrst}.
Важно подчеркнуть, что численное решение должно быть получено в постановке,
где левая и правая границы пласта совпадают с координатами торцов скважины,
а на этих границах задано условие непроницаемости.
Это необходимо для исключения нерадиального притока у торцов,
так как приведённое аналитическое решение выведено именно для такого случая.

Для оценки ошибки между численными и аналитическими величина введем обозначение
\begin{equation}
  \displaystyle
  E \left(\xi_{\text{an}}, \xi_{\text{num}}\right) \equiv E_{\xi} = \dfrac{ \left| \xi_{\text{num}} - \xi_{\text{an}} \right| }{\xi_{\text{num}}} \cdot 100,
  \label{eq:kham_common_residual}
\end{equation}
где $\xi$~--~это некоторый исследуемый параметр.

\begin{figure}[!ht]
  \centering
  \includesvg[width=0.5\linewidth]{images/hw/hw_inner/kham_hw_inner_theta.svg}
  \caption{Поправочный коэффициент $\theta$}
  \label{fig:kham_hw_inner_theta_map}
\end{figure}

Из рис.~\ref{fig:kham_hw_inner_theta_map} видно, что величина поправочного коэффициента $\theta$ не зависит от $r_w$ и для любых $H$ остаётся близкой к единице.
Действительно, при $\theta=1$ невязка~(\ref{eq:kham_common_residual}) по давлению $p_{\Gamma}$ варьируется от 2 до 9~\% (рис.~\ref{fig:kham_hw_inner_epg_map}), а по дебиту $q$ не превышает 4~\% (рис.~\ref{fig:kham_hw_inner_eq_map}).

Поэтому в дальнейших расчетах будет считать, что в формуле~(\ref{eq:kham_hw_inner_pg}) можно положить $\theta=1$.

\begin{figure}[!ht]
  \centering
  \begin{subfigure}{0.48\textwidth}
    \centering
    \includesvg[width=\linewidth]{images/hw/hw_inner/kham_hw_inner_Epg.svg}
    \caption{}
    \label{fig:kham_hw_inner_epg_map}
  \end{subfigure}
  \hfill
  \begin{subfigure}{0.48\textwidth}
    \centering
    \includesvg[width=\linewidth]{images/hw/hw_inner/kham_hw_inner_Eq.svg}
    \caption{}
    \label{fig:kham_hw_inner_eq_map}
  \end{subfigure}
  \caption{
    Зависимости невязки от $H$, $r_w$ (при $\theta=1$):
  а)~$E_{p_{\Gamma}}$
б)~$E_q$;
}
\label{fig:kham_hw_inner_theta_epg_maps}
\end{figure}

Численные расчеты выполнены в MRST на прямоугольной сетке со следующими параметрами : $nx=  135, ny = 3, nz = 13$.

Аналитическое и численное давление на границе $\Gamma$ тем лучше согласуются, чем меньше радиус скважины $r_w$ и больше высота пласта $H$ (рис.~\ref{fig:kham_hw_inner_epg_map}). Максимальное отклонение наблюдается при $H = 0.05$, $r = 0.002$, минимальное~--~при $H = 0.25$, $r = 0.0005$. Распределения давления в этих случаях приведены на рис.~\ref{fig:kham_hw_inner_press_disrt}.

\begin{figure}[!ht]
\centering
\begin{subfigure}{0.48\textwidth}
\centering
\includesvg[width=\linewidth]{images/hw/hw_inner/kham_hw_inner_p_H005_rw0002.svg}
\caption{}
\label{fig:kham_hw_inner_p_worst_pg}
\end{subfigure}
\hfill
\begin{subfigure}{0.48\textwidth}
\centering
\includesvg[width=\linewidth]{images/hw/hw_inner/kham_hw_inner_p_H025_rw00005.svg}
\caption{}
\label{fig:kham_hw_inner_p_best_pg}
\end{subfigure}
\caption{
Распределение давления в пласте \\ (сплошная линия~--~аналитическое решение, маркеры~--~численное ): \\
а)~$H = 0.05$, $r = 0.002$;
б)~$H = 0.25$, $r = 0.0005$
}
\label{fig:kham_hw_inner_press_disrt}
\end{figure}

В таблице \ref{tab:kham_hw_inner_p_error_metrics} представлены метрики ошибки для указанных двух наборов параметров.

\begin{table}[h!]
\centering
\caption{Сравнение аналитического и численного давления $p$ при разных параметрах}
\label{tab:kham_hw_inner_p_error_metrics}
\begin{tabular}{|c|c|c|c|}
\hline
\textbf{Параметры} & \textbf{RMSE} & \textbf{Mean} & \textbf{Std} \\
\hline
$H=0.05, r=0.002$ & $7.73 \cdot 10^{-3}$ & $6.11 \cdot 10^{-3}$ & $4.78 \cdot 10^{-3}$ \\
\hline
$H=0.25, r=0.0005$ & $3.78 \cdot 10^{-2}$ & $1.28 \cdot 10^{-2}$ & $3.58 \cdot 10^{-2}$ \\
\hline
\end{tabular}
\end{table}

Как видно из таблицы \ref{tab:kham_hw_inner_p_error_metrics}, конфигурация  с б\'{o}льшим отклонение $E_{p_{\Gamma}}$ демонстрирует на порядок лучшие значения метрик. Это связано, во-первых, с тем, что в обоих случаях ошибки являются малыми (относительно масштаба давления, принятого за 1). Во-вторых, можно предположить, что максимальное отклонение наблюдается вблизи скважины, а не на границе $\Gamma$, что приводит к более высокой вычислительной ошибке численного решения в этой области для малых $r_w$.

% ====================== ТОРЕЦ СКВАЖИНЫ ===========================
\subsubsection{Торец скважины}
Плоско-радиальная симметрия фильтрационного течения нарушается вблизи концов ствола скважины, для которых характерным является приток из внешней полусферы.
Для учёта геометрии у торца изменим аппроксимацию внешнего контура: вместо
прямоугольника рассматриваем прямоугольник с закруглёнными углами
радиуса 1~(рис.~\ref{fig:kham_well_end_sch}).
Течение жидкости между двумя вертикальными цилиндрическими поверхностями радиусов $H$ и 1, соответственно, имеет плоско-радиальный характер. Будем предполагать, что давление на внутренней цилиндрической поверхности радиуса $H$ незначительно отличается от давления на поверхности полусферы такого же радиуса.

\begin{figure}[!ht]
\centering
\includesvg[width=0.6\linewidth]{images/schemes/kham_hw_outer_scheme.svg}
\caption{Область решения задачи о притоке жидкости к торцу ствола ГС}
\label{fig:kham_well_end_sch}
\end{figure}

В области (I), ограниченной двумя полусферами радиусов $r_w$ и $H$, характер течения подчиняется сферической симметрии
\begin{equation}
\displaystyle
q^I = \theta \dfrac{\pi p_{\Gamma} H r_w}{\left(H - r_w\right)} , \;\;\;
p^I(r) = p_{\Gamma}\dfrac{\left(r-r_w\right)}{\left(H-r_w\right)}\dfrac{H}{r},
\label{eq:kham_hw_outer_spheric}
\end{equation}
где $q^I$~--~удельный приток флюида из четверти внешней полусферы. Эта величина полностью определяется внешним плоско-радиальным течением в области (II), $H \le r \le 1$, с граничными условиями $p(H) = p_{\Gamma}$, $p(1)= 1$ и по аналогии с (\ref{eq:kham_hw_inner_rad_qp}), равна
\begin{equation}
\displaystyle
q^{(II)} = \dfrac{\pi H \left( p_{\Gamma} - 1 \right)}{ \ln{H }}, \;\;\;
p^{(II)}(r) = 1 - \left(1-p_{\Gamma}\right) \dfrac{\ln{r}}{\ln{H}}
\label{eq:kham_hw_outer_cyl}
\end{equation}

Приравняв значения расходов (\ref{eq:kham_hw_outer_spheric}) и (\ref{eq:kham_hw_outer_cyl}), получим выражение для давления $p^R_{\Gamma}$ на границе $\Gamma$ вблизи торца скважины:
\begin{equation}
\displaystyle
p_{\Gamma} = \left(1- \theta \dfrac{r_w \ln{H}}{H - r_w}\right)^{-1}
\label{eq:kham_hw_outer_pg}
\end{equation}
Приток к торцу скважины определяется по формуле (\ref{eq:kham_hw_outer_spheric})
\begin{equation}
\displaystyle
Q_w^E = \dfrac{2\pi p_{\Gamma} H r_w}{\left(H - r_w\right)}
\label{eq:kham_hw_outer_Q}
\end{equation}
Поправочный коэффициент $\theta$ вычисляем из условия равенства аналитического $Q_w^E$ и численного $Q_{\text{num}}$ дебитов.
Тогда из~(\ref{eq:kham_hw_outer_Q}) и (\ref{eq:kham_hw_outer_pg}) следует:
\begin{equation}
\displaystyle
\theta = \dfrac{Q_{\text{num}} \left(H - r_w \right) - 2 \pi H r_w}{r_w Q_{\text{num}} \ln{H}}
\label{eq:kham_hw_outer_theta}
\end{equation}

Для расчёта сферического притока в MRST рассмотрим квадратную область в
плоскости $xy$, в центре которой расположена скважина радиуса $r_w$. В MRST перфорация задаётся по ячейкам, поэтому длина скважины играет второстепенную роль;
важно лишь, чтобы вся скважина целиком находилась внутри одной ячейки. Сетка строится следующим образом: центральная ячейка выбирается так, чтобы её размер
был кратен $r_w$, то есть
\begin{equation*}
\displaystyle
d_{\text{min}} = C \cdot r_w.
\end{equation*}
Далее во всех трёх направлениях сетка генерируется по сферическому закону~(рис.~\ref{fig:kham_hw_outer_grid_mrst}). Коэффициент $C$ подбирается из условия сходимости решение относительно дебита скважины. К сожалению, достичь строгой сходимости не удалось~(рис.~\ref{fig:kham_hw_outer_q_C_conv_MRST}). Минимальный $C$, при котором MRST корректно работает, оказалось равным 6; при меньших значениях возникает ошибка из-за переполнения памяти.

Ясно, что в MRST отсутствует встроенный учёт сферической симметрии.
Поэтому даже при использовании сеток, сгущающихся к торцу скважины,
уловить тонкие особенности сферического притока не удаётся~(рис.~\ref{fig:kham_hw_outer_grid_mrst}).
\begin{figure}[!ht]
\centering
\begin{subfigure}{0.48\textwidth}
\centering
\includesvg[width=\linewidth]{images/hw/hw_outer/kham_hw_outer_p_field_mrst.svg}
\caption{}
\label{fig:kham_hw_outer_grid_mrst_3D}
\end{subfigure}
\hfill
\begin{subfigure}{0.48\textwidth}
\centering
\includesvg[width=\linewidth]{images/hw/hw_outer/kham_hw_outer_p_field_xz_mrst.svg}
\caption{}
\label{fig:kham_hw_outer_grid_mrst_xz}
\end{subfigure}
\caption{
Сетка и поле давления у торца скважины (MRST): \\
а)~в 3D;
б)~вдоль линии $y=100$ \\
(на рисунке все значения в размерном виде)
}
\label{fig:kham_hw_outer_grid_mrst}
\end{figure}

\begin{figure}[!ht]
\centering
\includesvg[width=0.5\linewidth]{images/hw/hw_outer/kham_hw_outer_q_C_conv_mrst.svg}
\caption{\label{fig:kham_hw_outer_q_C_conv_MRST}Сходимость численного решения MRST в зависимости от минимального размера ячейки около скважины}
\end{figure}

\begin{figure}[!ht]
\centering
\includesvg[width=0.5\linewidth]{images/hw/hw_outer/kham_hw_outer_theta_mrst.svg}
\caption{Поправочный коэффициента $\theta$ для задачи на торце скважины (MRST)}
\label{fig:kham_hw_outer_theta_map_mrst}
\end{figure}

На рис.~\ref{fig:kham_hw_outer_theta_map_mrst} показаны значения поправочного коэффициента $\theta$ для торца скважины. В отличие от случая основного ствола, здесь значения $\theta$ оказываются менее устойчивыми и зависят как от $H$, так и от $r_w$.

\begin{figure}[!ht]
\centering
\begin{subfigure}{0.48\textwidth}
\centering
\includesvg[width=\linewidth]{images/hw/hw_outer/kham_hw_outer_Eq_mrst.svg}
\caption{}
\label{fig:kham_hw_outer_eq_map_mrst}
\end{subfigure}
\hfill
\begin{subfigure}{0.48\textwidth}
\centering
\includesvg[width=\linewidth]{images/hw/hw_outer/kham_hw_outer_Epg_mrst.svg}
\caption{}
\label{fig:kham_hw_outer_epg_map_mrst}
\end{subfigure}
\caption{
Зависимости невязки от $H$, $r_w$ (MRST, при $\theta = 1$):
а)~$E_q$;
б)~$E_{p_{\Gamma}}$
}
\label{fig:kham_hw_outer_eq_epg_maps_mrst}
\end{figure}

Из рис.~\ref{fig:kham_hw_outer_eq_epg_maps_mrst} видно, что невязка по давлению
$p_{\Gamma}$ достигает значений порядка единицы,
а ошибка по дебиту $q$ также оказывается значительной.
Это указывает на то, что сферический характер притока у торца скважины в MRST воспроизводится плохо.

\begin{figure}[!ht]
\centering
\begin{subfigure}{0.48\textwidth}
\centering
\includesvg[width=\linewidth]{images/hw/hw_outer/kham_hw_outer_p_H005_rw0002_mrst.svg}
\caption{}
\label{fig:kham_hw_outer_p_worst_pg_mrst}
\end{subfigure}
\hfill
\begin{subfigure}{0.48\textwidth}
\centering
\includesvg[width=\linewidth]{images/hw/hw_outer/kham_hw_outer_p_H025_rw00005_mrst.svg}
\caption{}
\label{fig:kham_hw_outer_p_best_pg_mrst}
\end{subfigure}
\caption{
Распределение давления в пласте (MRST) \\
(сплошная линия~--~аналитическое решение, маркеры~--~численное ): \\
а)~$H = 0.05$, $r = 0.002$;
б)~$H = 0.25$, $r = 0.0005$
}
\label{fig:kham_hw_outer_press_disrt}
\end{figure}

На рис.~\ref{fig:kham_hw_outer_press_disrt} показаны характерные распределения давления. Даже для «лучших» параметров ($H=0.25$, $r=0.0005$) наблюдаются заметные отклонения от аналитики.

\begin{table}[h!]
\centering
\caption{Сравнение аналитического и численного давления $p$ при разных параметрах (MRST)}
\label{tab:kham_hw_outer_p_error_metrics_mrst}
\begin{tabular}{|c|c|c|c|}
\hline
\textbf{Параметры} & \textbf{RMSE} & \textbf{Mean} & \textbf{Std} \\
\hline
$H=0.05, r=0.002$ & $0.62$ & $0.21$ & $0.59$ \\
\hline
$H=0.25, r=0.0005$ & $1.82$ & $0.39$ & $1.8$ \\
\hline
\end{tabular}
\end{table}

Данные таблицы~\ref{tab:kham_hw_outer_p_error_metrics_mrst} подтверждают вывод:
ошибки численного моделирования для области торца скважины велики,
и MRST оказывается недостаточно точным инструментом для анализа сферического притока.

Для корректно учета сферического течения на торцах скважины воспользуемся
численной моделью~\cite{lit:kham_mazo_uzku_2015}, в которой реализованы
специальные поправочные коэффициенты (радиальный и сферический) для вычисления нормальной скорости на поверхности скважины. Это позволяет обойти ограничения MRST, где сферическая симметрия не заложена и где численное решение оказалось нестабильным.

\begin{figure}[!ht]
\centering
\includesvg[width=0.5\linewidth]{images/hw/hw_outer/kham_hw_outer_theta_mgrp.svg}
\caption{Поправочный коэффициент $\theta$ для задачи на торце скважины (по модели~\cite{lit:kham_mazo_uzku_2015})}
\label{fig:kham_hw_outer_theta_map_mgrp}
\end{figure}

Как видно из рис.~\ref{fig:kham_hw_outer_theta_map_mgrp},
учёт сферического притока через поправочные коэффициенты обеспечивает
устойчивое поведение $\theta$ и согласование с аналитическими решениями.
Таким образом, для описания притока у торцов скважины более целесообразно
опираться на данную модель, чем на моделирование в MRST.

На рис.~\ref{fig:kham_hw_outer_theta_map_mgrp} показано, что $\theta$ принимает значения от 1.1 до 2.4, но относительная ошибка~(\ref{eq:kham_common_residual}) по дебиту $Q$ не превышает 1.1\% (рис.~\ref{fig:kham_hw_outer_eq_map_mgrp}). Это связано с тем, что $\theta$ чрезвычайно чувствительная к параметрам $H$, $r_w$. Поэтому даже небольшие изменения дебита вызывают значительные колебания значения $\theta$. Следовательно, даже если $\theta$ превышает единицу в несколько раз, в формуле~(\ref{eq:kham_hw_outer_pg}) можно положить $\theta=1$, и при этом ошибка в дебите не будет превышать одного процента~(рис.~\ref{fig:kham_hw_outer_eq_map_mgrp}).

\begin{figure}[!ht]
\centering
\begin{subfigure}{0.48\textwidth}
\centering
\includesvg[width=\linewidth]{images/hw/hw_outer/kham_hw_outer_Eq_mgrp.svg}
\caption{}
\label{fig:kham_hw_outer_eq_map_mgrp}
\end{subfigure}
\hfill
\begin{subfigure}{0.48\textwidth}
\centering
\includesvg[width=\linewidth]{images/hw/hw_outer/kham_hw_outer_Epg_mgrp.svg}
\caption{}
\label{fig:kham_hw_outer_epg_map_mgrp}
\end{subfigure}
\caption{
Зависимости от $H$, $r_w$ (по модели~\cite{lit:kham_mazo_uzku_2015}, при $\theta = 1$):
а)~$E_q$;
б)~$E_{p_{\Gamma}}$
}
\label{fig:kham_hw_outer_theta_epg_maps_mgrp}
\end{figure}

Для оценки влияния поправочного коэффициента $\theta$ на точность расчёта удобно ввести величину
\[
\alpha = \frac{r_w \ln H}{H-r_w}.
\]

Тогда относительная ошибка $E_Q$~(\ref{eq:kham_common_residual}) между численным дебитом $Q_{\text{num}}$ и аналитическим выражением $Q_w^E$ имеет вид
\begin{equation}
E_Q = \dfrac{\alpha(\theta-1)}{1-\alpha\theta}.
\label{eq:kham_theta_error_exact}
\end{equation}

Для оценки границ, когда $\theta$ может быть задана как единица в формуле~(\ref{eq:kham_hw_outer_pg}), можно воспользоваться условием
\begin{equation*}
|E_Q| \leq \varepsilon,
\end{equation*}
где $\varepsilon$~--- допустимая относительная ошибка в процентах.

Как видно из рис.~\ref{fig:kham_hw_outer_eq_map_mgrp}, при $\theta=1$
в формуле~(\ref{eq:kham_hw_outer_pg}) относительная погрешность не превышает
1.1\% во всём диапазоне изменения $H$ и $r_w$. При этом погрешность тем
больше, чем меньше $H$ и больше $r_w$. Это связано с тем, что при малой
толщине пласта и относительно крупном радиусе скважины увеличивается влияние
локальной сферической зоны около торца, а приближение $\theta=1$ перестаёт
в полной мере компенсировать различие между численной и аналитической
моделями. Тем не менее, величина ошибки остаётся существенно меньше
погрешностей практических расчётов, поэтому для инженерных оценок можно
уверенно полагать $\theta=1$.

\begin{figure}[!ht]
\centering
\begin{subfigure}{0.48\textwidth}
\centering
\includesvg[width=\linewidth]{images/hw/hw_outer/kham_hw_outer_p_H005_rw0002_mgrp.svg}
\caption{}
\label{fig:kham_hw_outer_p_worst_pg_mgrp}
\end{subfigure}
\hfill
\begin{subfigure}{0.48\textwidth}
\centering
\includesvg[width=\linewidth]{images/hw/hw_outer/kham_hw_outer_p_H025_rw00005_mgrp.svg}
\caption{}
\label{fig:kham_hw_outer_p_best_pg_mgrp}
\end{subfigure}
\caption{
Распределение давления в пласте (по модели~\cite{lit:kham_mazo_uzku_2015}) \\
(сплошная линия~--~аналитическое решение, маркеры~--~численное ): \\
а)~$H = 0.05$, $r = 0.002$;
б)~$H = 0.25$, $r = 0.0005$
}
\label{fig:kham_hw_outer_press_disrt_mgrp}
\end{figure}

На рис.~\ref{fig:kham_hw_outer_press_disrt_mgrp} показано распределение давления в пласте, вычисленное по модели~\cite{lit:kham_mazo_uzku_2015}.
Наблюдается хорошая согласованность численного и аналитического решений. Это подтверждается также данными из таблицы~\ref{tab:kham_hw_outer_p_error_metrics_mgrp}.

\begin{table}[h!]
\centering
\caption{Сравнение аналитического и численного давления $p$ при разных параметрах (по модели~\cite{lit:kham_mazo_uzku_2015})}
\label{tab:kham_hw_outer_p_error_metrics_mgrp}
\begin{tabular}{|c|c|c|c|}
\hline
\textbf{Параметры} & \textbf{RMSE} & \textbf{Mean} & \textbf{Std} \\
\hline
$H=0.05, r=0.002$ & $2.81 \cdot 10^{-3}$ & $1.07 \cdot 10^{-3}$ & $2.61 \cdot 10^{-3}$ \\
\hline
$H=0.25, r=0.0005$ & $1.13 \cdot 10^{-3}$ & $1.51 \cdot 10^{-3}$ & $1.13 \cdot 10^{-3}$ \\
\hline
\end{tabular}
\end{table}

\subsubsection{Доля сферического притока}

Оценим долю притока с торцов скважины.

На рис.~\ref{fig:kham_hw_q_sph_part_an} приведены поля $Q_w^E / Q_w^P$, вычисленные по аналитическим формулам
(\ref{eq:kham_hw_inner_Q}) и (\ref{eq:kham_hw_outer_Q}) (на самом деле тут $2 Q_w^E$, так как торцов два).
Ясно, что для толстой скважины и/или тонкого пласта эффект от сферического притока будет более существенным, порядка 20~\%
для коротких скважин $L < 1$. Однако, данный эффект нивелируется для протяженных скважин $L \geq 5$.

\begin{figure}[h!]
\centering
% --- Первый ряд ---
\begin{subfigure}{0.48\textwidth}
\includesvg[width=\linewidth]{images/hw/rad_sph_partition/kham_q_sph_part_L0.5_an.svg}
\caption{}
\label{fig:kham_hw_qsph_qan_an_L05}
\end{subfigure}
\hfill
\begin{subfigure}{0.48\textwidth}
\includesvg[width=\linewidth]{images/hw/rad_sph_partition/kham_q_sph_part_L0.7_an.svg}
\caption{}
\label{fig:kham_hw_qsph_qan_an_L07}
\end{subfigure}

% --- Второй ряд ---
\vskip\baselineskip
\begin{subfigure}{0.48\textwidth}
\includesvg[width=\linewidth]{images/hw/rad_sph_partition/kham_q_sph_part_L1_an.svg}
\caption{}
\label{fig:kham_hw_qsph_qan_an_L1}
\end{subfigure}
\hfill
\begin{subfigure}{0.48\textwidth}
\includesvg[width=\linewidth]{images/hw/rad_sph_partition/kham_q_sph_part_L5_an.svg}
\caption{}
\label{fig:kham_hw_qsph_qan_an_L5}
\end{subfigure}

\caption{Доля сферического притока (аналитический расчет): \\ а)~$L$=0.5, б)~$L$=0.7, в)~$L$=1.0, г)~$L$=5.0}
\label{fig:kham_hw_q_sph_part_an}
\end{figure}

По сути столь короткие скважины $L < 1$ на практике не встречаются и можно сделать вывод, что сферическим притоком можно в большинстве случаем пренебречь.

Однако, если взглянуть на такие же карты, но полученные из численной модели~(\cite{lit:kham_mazo_uzku_2015}), то выводы будут другие.

\begin{figure}[h!]
\centering
% --- Первый ряд ---
\begin{subfigure}{0.48\textwidth}
\includesvg[width=\linewidth]{images/hw/rad_sph_partition/kham_q_sph_part_L0.5_num.svg}
\caption{}
\label{fig:kham_hw_qsph_qan_num_L05}
\end{subfigure}
\hfill
\begin{subfigure}{0.48\textwidth}
\includesvg[width=\linewidth]{images/hw/rad_sph_partition/kham_q_sph_part_L0.7_num.svg}
\caption{}
\label{fig:kham_hw_qsph_qan_num_L07}
\end{subfigure}

% --- Второй ряд ---
\vskip\baselineskip
\begin{subfigure}{0.48\textwidth}
\includesvg[width=\linewidth]{images/hw/rad_sph_partition/kham_q_sph_part_L1_num.svg}
\caption{}
\label{fig:kham_hw_qsph_qan_num_L1}
\end{subfigure}
\hfill
\begin{subfigure}{0.48\textwidth}
\includesvg[width=\linewidth]{images/hw/rad_sph_partition/kham_q_sph_part_L5_num.svg}
\caption{}
\label{fig:kham_hw_qsph_qan_num_L5}
\end{subfigure}

\caption{Доля сферического притока (численный~\cite{lit:kham_mazo_uzku_2015} расчет): \\ а)~$L$=0.5, б)~$L$=0.7, в)~$L$=1.0, г)~$L$=5.0}
\label{fig:kham_hw_q_sph_part_num}
\end{figure}

Из рис.~\ref{fig:kham_hw_q_sph_part_num} видно, что доля притока к торцам
пренебрежимо мала (сотые доли процентов) по сравнению с притоком в основной ствол скважины и результаты,
полученные по аналитической модели, существенно
отличаются от численных. Это связано с тем, что вблизи торцов характер
течения меняется: радиальный приток трансформируется в смешанный режим,
и аналитическая аппроксимация уже не описывает его корректно.
Данный эффект подробно рассматривается в следующей подглаве.

\subsubsection{Проверка гипотезы о нарушении радиального притока вблизи торцов скважины}

В дальнейшем будем снова рассматривать внешний контур питания в виде прямоугольной
области. Такой выбор оправдан,
поскольку сферический приток у торцов более не учитывается, а интерес
представляет именно распределение радиального притока вдоль длины скважины.

Как мы отметили выше, радиальный приток к стволу скважины нарушается вблизи его торцов, где преобладает сферический приток. Чтобы это проверить решим численно следующую тестовую задачу: возьмем скважины различной длины и построим карты $E_p$~(\ref{eq:kham_common_residual}) от $H, r_w$ для сферического и радиального притока по ортогональным направлениям вдоль ствола скважины.

\begin{figure}[h!]
\centering
% --- Первый ряд ---
\begin{subfigure}{0.3\textwidth}
\includesvg[width=\linewidth]{images/hw/rad_hypo_fault/kham_well_along_Ep_H0.05_L0.5.svg}
\caption{}
\label{fig:kham_rad_hypo_fault_p_over_l_H005_L05}
\end{subfigure}
\hfill
\begin{subfigure}{0.3\textwidth}
\includesvg[width=\linewidth]{images/hw/rad_hypo_fault/kham_well_along_Ep_H0.05_L1.svg}
\caption{}
\label{fig:kham_rad_hypo_fault_p_over_l_H005_L1}
\end{subfigure}
\hfill
\begin{subfigure}{0.3\textwidth}
\includesvg[width=\linewidth]{images/hw/rad_hypo_fault/kham_well_along_Ep_H0.05_L5.svg}
\caption{}
\label{fig:kham_rad_hypo_fault_p_over_l_H005_L5}
\end{subfigure}

% --- Второй ряд ---
\begin{subfigure}{0.3\textwidth}
\includesvg[width=\linewidth]{images/hw/rad_hypo_fault/kham_well_along_Ep_H0.15_L0.5.svg}
\caption{}
\label{fig:kham_rad_hypo_fault_p_over_l_H015_L05}
\end{subfigure}
\hfill
\begin{subfigure}{0.3\textwidth}
\includesvg[width=\linewidth]{images/hw/rad_hypo_fault/kham_well_along_Ep_H0.15_L1.svg}
\caption{}
\label{fig:kham_rad_hypo_fault_p_over_l_H015_L1}
\end{subfigure}
\hfill
\begin{subfigure}{0.3\textwidth}
\includesvg[width=\linewidth]{images/hw/rad_hypo_fault/kham_well_along_Ep_H0.15_L5.svg}
\caption{}
\label{fig:kham_rad_hypo_fault_p_over_l_H015_L5}
\end{subfigure}

% --- Third row ---
\begin{subfigure}{0.3\textwidth}
\includesvg[width=\linewidth]{images/hw/rad_hypo_fault/kham_well_along_Ep_H0.25_L0.5.svg}
\caption{}
\label{fig:kham_rad_hypo_fault_p_over_l_H025_L05}
\end{subfigure}
\hfill
\begin{subfigure}{0.3\textwidth}
\includesvg[width=\linewidth]{images/hw/rad_hypo_fault/kham_well_along_Ep_H0.25_L1.svg}
\caption{}
\label{fig:kham_rad_hypo_fault_p_over_l_H025_L1}
\end{subfigure}
\hfill
\begin{subfigure}{0.3\textwidth}
\includesvg[width=\linewidth]{images/hw/rad_hypo_fault/kham_well_along_Ep_H0.25_L5.svg}
\caption{}
\label{fig:kham_rad_hypo_fault_p_over_l_H025_L5}
\end{subfigure}

\caption{$E_p$ вдоль ствола скважины (красные линии~--~относительно сферического распределения давления,
синии~--~относительно радиального)}
\label{fig:kham_rad_hypo_fault_p_over_l}
\end{figure}

На рис.~\ref{kham_rad_hypo_fault_p_over_l_H015_rw00001_L05} показано распределение давление в
в ортогональных к скважине линиях для различных относительных координат вдоль ствола при $H=0.15$, $r_w = 0.0001$.

\begin{figure}[h!]
\centering
\begin{subfigure}{0.3\textwidth}
\includesvg[width=\linewidth]{images/hw/rad_hypo_fault/kham_py_L05_rw0001_H015_mgrp.svg}
\caption{}
\label{fig:kham_py_L05_rw0001_H015_mgrp}
\end{subfigure}
\hfill
\begin{subfigure}{0.3\textwidth}
\includesvg[width=\linewidth]{images/hw/rad_hypo_fault/kham_py_L2_rw0001_H015_mgrp.svg}
\caption{}
\label{fig:kham_py_L2_rw0001_H015_mgrp}
\end{subfigure}
\hfill
\begin{subfigure}{0.3\textwidth}
\includesvg[width=\linewidth]{images/hw/rad_hypo_fault/kham_py_L10_rw0001_H015_mgrp.svg}
\caption{}
\label{fig:kham_py_L10_rw0001_H015_mgrp}
\end{subfigure}

\caption{Давление в ортогональных к скважине линиях для различных относительных координат вдоль ствола
($H=0.15$, $r_w = 0.0001$) : а)~$L=0.5$; б)~$L=2$; в)~$L=10$} \\
(пунктирная линия~--~аналитическое радиальное распределение давления, штрих-пунктирное~--~сферическое)
\label{fig:kham_rad_hypo_fault_p_over_l_H015_rw00001_L05}
\end{figure}

Как видно из рис.~\ref{fig:kham_rad_hypo_fault_p_over_l} вблизи торцов скважины давление существенно отличается от радиального решения и тем более от сферического.
Это означает, что на всей протяжённости ствола скважины нельзя считать приток строго радиальным.
Однако высокая относительная ошибка в давлении не всегда означает, что ошибка будет высокой и для дебита.

\begin{figure}[h!]
\centering
% --- Первый ряд ---
\begin{subfigure}{0.3\textwidth}
\includesvg[width=\linewidth]{images/hw/rad_hypo_fault/kham_well_along_Eq_H0.05_L0.5.svg}
\caption{}
\label{fig:kham_rad_hypo_fault_q_l_H005_L05}
\end{subfigure}
\hfill
\begin{subfigure}{0.3\textwidth}
\includesvg[width=\linewidth]{images/hw/rad_hypo_fault/kham_well_along_Eq_H0.05_L1.svg}
\caption{}
\label{fig:kham_rad_hypo_fault_q_l_H005_L1}
\end{subfigure}
\hfill
\begin{subfigure}{0.3\textwidth}
\includesvg[width=\linewidth]{images/hw/rad_hypo_fault/kham_well_along_Eq_H0.05_L5.svg}
\caption{}
\label{fig:kham_rad_hypo_fault_q_l_H005_L5}
\end{subfigure}

% --- Второй ряд ---
\begin{subfigure}{0.3\textwidth}
\includesvg[width=\linewidth]{images/hw/rad_hypo_fault/kham_well_along_Eq_H0.15_L0.5.svg}
\caption{}
\label{fig:kham_rad_hypo_fault_q_l_H015_L05}
\end{subfigure}
\hfill
\begin{subfigure}{0.3\textwidth}
\includesvg[width=\linewidth]{images/hw/rad_hypo_fault/kham_well_along_Eq_H0.15_L1.svg}
\caption{}
\label{fig:kham_rad_hypo_fault_q_l_H015_L1}
\end{subfigure}
\hfill
\begin{subfigure}{0.3\textwidth}
\includesvg[width=\linewidth]{images/hw/rad_hypo_fault/kham_well_along_Eq_H0.15_L5.svg}
\caption{}
\label{fig:kham_rad_hypo_fault_q_l_H015_L5}
\end{subfigure}

% --- Third row ---
\begin{subfigure}{0.3\textwidth}
\includesvg[width=\linewidth]{images/hw/rad_hypo_fault/kham_well_along_Eq_H0.25_L0.5.svg}
\caption{}
\label{fig:kham_rad_hypo_fault_q_l_H025_L05}
\end{subfigure}
\hfill
\begin{subfigure}{0.3\textwidth}
\includesvg[width=\linewidth]{images/hw/rad_hypo_fault/kham_well_along_Eq_H0.25_L1.svg}
\caption{}
\label{fig:kham_rad_hypo_fault_q_l_H025_L1}
\end{subfigure}
\hfill
\begin{subfigure}{0.3\textwidth}
\includesvg[width=\linewidth]{images/hw/rad_hypo_fault/kham_well_along_Eq_H0.25_L5.svg}
\caption{}
\label{fig:kham_rad_hypo_fault_q_l_H025_L5}
\end{subfigure}

\caption{
Удельный приток $q$ вдоль ствола скважины (расчеты сделаны по модели~\cite{lit:kham_mazo_uzku_2015});
\\ (сплошная линия~--~численное решение, пунктирная~--~аналитическое)
}
\label{fig:kham_rad_hypo_fault_q_l}
\end{figure}

Рассмотрим подробнее вариант $H=0.15$ (рис.~\ref{fig:kham_rad_hypo_fault_q_l_constH015}).
Видно, что для длинных скважин удельный приток к середине падает ниже
аналитического~(рис.~\ref{fig:kham_rad_hypo_fault_q_l_H025_L5}).
Чтобы убедиться, что проблема не в численном решение, произведем те же вычисления в MRST.

\begin{figure}[h!]
\centering
% --- Первый ряд ---
\begin{subfigure}{0.45\textwidth}
\includesvg[width=\linewidth]{images/hw/rad_hypo_fault/kham_well_along_Eq_H0.15_L0.5.svg}
\caption{}
\label{fig:kham_rad_hypo_fault_q_l_constH015_L05}
\end{subfigure}
\hfill
\begin{subfigure}{0.45\textwidth}
\includesvg[width=\linewidth]{images/hw/rad_hypo_fault/kham_well_along_Eq_H0.15_L1.svg}
\caption{}
\label{fig:kham_rad_hypo_fault_q_l_constH015_L1}
\end{subfigure}

% --- Второй ряд ---
\begin{subfigure}{0.45\textwidth}
\includesvg[width=\linewidth]{images/hw/rad_hypo_fault/kham_well_along_Eq_H0.15_L2.svg}
\caption{}
\label{fig:kham_rad_hypo_fault_q_l_constH015_L2}
\end{subfigure}
\hfill
\begin{subfigure}{0.45\textwidth}
\includesvg[width=\linewidth]{images/hw/rad_hypo_fault/kham_well_along_Eq_H0.15_L3.svg}
\caption{}
\label{fig:kham_rad_hypo_fault_q_l_constH015_L3}
\end{subfigure}

% --- Third row ---
\begin{subfigure}{0.45\textwidth}
\includesvg[width=\linewidth]{images/hw/rad_hypo_fault/kham_well_along_Eq_H0.15_L5.svg}
\caption{}
\label{fig:kham_rad_hypo_fault_q_l_constH015_L5}
\end{subfigure}
\hfill
\begin{subfigure}{0.45\textwidth}
\includesvg[width=\linewidth]{images/hw/rad_hypo_fault/kham_well_along_Eq_H0.15_L10.svg}
\caption{}
\label{fig:kham_rad_hypo_fault_q_l_constH015_L10}
\end{subfigure}

\caption{
$E_q$ вдоль ствола скважины при $H=0.15$
(расчеты сделаны по модели~\cite{lit:kham_mazo_uzku_2015}); \\
(сплошная линия~--~численное решение, пунктирная~--~аналитическое)}
\label{fig:kham_rad_hypo_fault_q_l_constH015}
\end{figure}

\begin{figure}[h!]
\centering
% --- Первый ряд ---
\begin{subfigure}{0.3\textwidth}
\includesvg[width=\linewidth]{images/hw/rad_hypo_fault/kham_well_along_Eq_H0.05_L0.5_mrst.svg}
\caption{}
\label{fig:kham_rad_hypo_fault_q_l_H005_L05_mrst}
\end{subfigure}
\hfill
\begin{subfigure}{0.3\textwidth}
\includesvg[width=\linewidth]{images/hw/rad_hypo_fault/kham_well_along_Eq_H0.05_L1_mrst.svg}
\caption{}
\label{fig:kham_rad_hypo_fault_q_l_H005_L1_mrst}
\end{subfigure}
\hfill
\begin{subfigure}{0.3\textwidth}
\includesvg[width=\linewidth]{images/hw/rad_hypo_fault/kham_well_along_Eq_H0.05_L5_mrst.svg}
\caption{}
\label{fig:kham_rad_hypo_fault_q_l_H005_L5_mrst}
\end{subfigure}

% --- Второй ряд ---
\begin{subfigure}{0.3\textwidth}
\includesvg[width=\linewidth]{images/hw/rad_hypo_fault/kham_well_along_Eq_H0.15_L0.5_mrst.svg}
\caption{}
\label{fig:kham_rad_hypo_fault_q_l_H015_L05_mrst}
\end{subfigure}
\hfill
\begin{subfigure}{0.3\textwidth}
\includesvg[width=\linewidth]{images/hw/rad_hypo_fault/kham_well_along_Eq_H0.15_L1_mrst.svg}
\caption{}
\label{fig:kham_rad_hypo_fault_q_l_H015_L1_mrst}
\end{subfigure}
\hfill
\begin{subfigure}{0.3\textwidth}
\includesvg[width=\linewidth]{images/hw/rad_hypo_fault/kham_well_along_Eq_H0.15_L5_mrst.svg}
\caption{}
\label{fig:kham_rad_hypo_fault_q_l_H015_L5_mrst}
\end{subfigure}

% --- Third row ---
\begin{subfigure}{0.3\textwidth}
\includesvg[width=\linewidth]{images/hw/rad_hypo_fault/kham_well_along_Eq_H0.25_L0.5_mrst.svg}
\caption{}
\label{fig:kham_rad_hypo_fault_q_l_H025_L05_mrst}
\end{subfigure}
\hfill
\begin{subfigure}{0.3\textwidth}
\includesvg[width=\linewidth]{images/hw/rad_hypo_fault/kham_well_along_Eq_H0.25_L1_mrst.svg}
\caption{}
\label{fig:kham_rad_hypo_fault_q_l_H025_L1_mrst}
\end{subfigure}
\hfill
\begin{subfigure}{0.3\textwidth}
\includesvg[width=\linewidth]{images/hw/rad_hypo_fault/kham_well_along_Eq_H0.25_L5_mrst.svg}
\caption{}
\label{fig:kham_rad_hypo_fault_q_l_H025_L5_mrst}
\end{subfigure}

\caption{
$E_q$ вдоль ствола скважины (расчеты сделаны в MRST);
\\ (сплошная линия~--~численное решение, пунктирная~--~аналитическое)
}
\label{fig:kham_rad_hypo_fault_q_l_mrst}
\end{figure}

Как видим из рис.~\ref{fig:kham_rad_hypo_fault_q_l_mrst} удельный приток $q$ тоже дает значения ниже, чем по аналитической модели на удалении от торцов (по середине ствола скважины). Это говорит о том, что 1) наличие торцевого притока влияет на радиальный приток даже на отдалении от торцов (либо эта та самая разница в 3-4 \% (рис.~\ref{fig:kham_hw_inner_eq_map})); 2) численные решения в MRST и по модели~\cite{lit:kham_mazo_uzku_2015} хорошо согласуются (рис.~\ref{fig:kham_well_along_ql_H015_L5_rw0001_mgrp_mrst})

Оценим насколько формула $Q = Q_w^P \cdot L$ занижает дебит (сравним с численным решением).

\begin{figure}[h!]
\centering
% --- Первый ряд ---
\begin{subfigure}{0.48\textwidth}
\includesvg[width=\linewidth]{images/hw/rad_hypo_fault/kham_Rq_map_L05_mrst.svg}
\caption{}
\label{fig:kham_Rq_map_L0.5_mrst}
\end{subfigure}
\hfill
\begin{subfigure}{0.48\textwidth}
\includesvg[width=\linewidth]{images/hw/rad_hypo_fault/kham_Rq_map_L1_mrst.svg}
\caption{}
\label{fig:kham_Rq_map_L1_mrst}
\end{subfigure}

% --- Второй ряд ---
\begin{subfigure}{0.48\textwidth}
\includesvg[width=\linewidth]{images/hw/rad_hypo_fault/kham_Rq_map_L5_mrst.svg}
\caption{}
\label{fig:kham_Rq_map_L5_mrst}
\end{subfigure}
\hfill
\begin{subfigure}{0.48\textwidth}
\includesvg[width=\linewidth]{images/hw/rad_hypo_fault/kham_Rq_map_L10_mrst.svg}
\caption{}
\label{fig:kham_Rq_map_L10_mrst}
\end{subfigure}

\caption{
$E_q$ вдоль ствола скважины (расчеты сделаны в MRST);
\\ (сплошная линия~--~численное решение, пунктирная~--~аналитическое)
}
\label{fig:kham_Rq_map_mrst}
\end{figure}

\begin{figure}[!ht]
\centering
\includesvg[width=0.7\linewidth]{images/hw/kham_well_along_ql_mgrp_mrst_H015_L5_rw0001.svg}
\label{fig:kham_well_along_ql_H015_L5_rw0001_mgrp_mrst}
\caption{Удельный относительный приток к стволу скважины \\
сплошная линия~--~модель~\cite{lit:kham_mazo_uzku_2015}, маркеры~--~MRST)}
\end{figure}

\subsubsection{Полный приток к ГС}
Полный приток к горизонтальной скважине длины $L$ будет определяться по формуле
\begin{equation}
\displaystyle
Q = Q_w^P,
\label{eq:kham_hw_total}
\end{equation}
где $Q_w^P$~--~приток к основному стволу скважины~(\ref{eq:kham_qw_inner}).
При этом в формуле (\ref{eq:kham_hw_inner_pg}) берем $\theta=1$. Ограничения для пользования этой формулой $L \geq 5$. В таком случае отклонение от численного решения будет не более 10 \% (рис.~\ref{fig:kham_Rq_map_L5_mrst}).

\subsubsection{Сравнение с другими моделями}

\subsection{Приток к скважине с МГРП}

% Библиография
% \cleardoublepage % Начинаем с новой страницы (для двусторонней печати)
\phantomsection % Для гиперссылок в PDF
\bibliographystyle{ugost2008}
%\bibliographystyle{ugost2008ls} % Стиль с сортировкой по упоминанию % или \bibliographystyle{ugost2008} % Алфавитная сортировка
\bibliography{references}
\addcontentsline{toc}{section}{Список литературы}

\end{document}
